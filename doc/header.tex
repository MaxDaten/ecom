% Einseitig A4, 12 Punkt Schriftgröße, Literatur-, Tabellen- und Abbildungsverzeichnis im Inhaltsverzeichnis
% Kopf- und Fußzeile zählen zum Rand und nicht zum Textkörper, Bindekorrektur von 5 mm, zwischen den Absätzen
% 0.5 zeiligen Abstand
\documentclass[a4paper,12pt,headsepline,bibtotoc,liststotoc,headexclude,footexclude,BCOR5mm,halfparskip]{scrreprt} %draft

% Umlaute können direkt als Umlaute geschrieben werden
\usepackage[utf8x]{inputenc}

% Silbentrennung etc. in neuer deutscher Rechtschreibung
\usepackage[ngerman]{babel}

% Zusätzliche Symbole
\usepackage{textcomp}

% Schickerer Satzspiegel mit KOMA-Script
\usepackage{scrpage2}

% Zum setzen des Zeilenabstands
\usepackage{setspace}\usepackage{listings}
\lstloadlanguages{Haskell}
\lstnewenvironment{code}
    {\lstset{}%
      \csname lst@SetFirstLabel\endcsname}
    {\csname lst@SaveFirstLabel\endcsname}
    \lstset{
      basicstyle=\small\ttfamily,
      flexiblecolumns=false,
      basewidth={0.5em,0.45em},
      literate={+}{{$+$}}1 {/}{{$/$}}1 {*}{{$*$}}1 {=}{{$=$}}1
               {>}{{$>$}}1 {<}{{$<$}}1 {\\}{{$\lambda$}}1
               {\\\\}{{\char`\\\char`\\}}1
               {->}{{$\rightarrow$}}2 {>=}{{$\geq$}}2 {<-}{{$\leftarrow$}}2
               {<=}{{$\leq$}}2 {=>}{{$\Rightarrow$}}2 
               {\ .}{{$\circ$}}2 {\ .\ }{{$\circ$}}2
               {>>}{{>>}}2 {>>=}{{>>=}}2
               {|}{{$\mid$}}1               
    }

% Für Grafiken
\usepackage{graphicx}

% Rahmen in 0Pixel Entfernung zeichnen
\setlength{\fboxsep}{0pt}

% Tabellen mit automatischem Zeilenumbruch und multirow
\usepackage{tabularx}
\usepackage{multirow}

% Für textumflossene Bilder
\usepackage{float}
\usepackage{wrapfig}

% Für Listings
\usepackage{listings}

% Für Tabellen mit multirow
\usepackage{multirow}

% Für Formeln
\usepackage{amsmath}
\usepackage{amsfonts}
\usepackage{amssymb}
\usepackage{amsthm}
\usepackage[german, intoc]{nomencl}

% Satzspiegel neu berechnen
\typearea[current]{12}

% Volle Seite verwenden
%\usepackage{fullpage}

% Standardseiten mit Kopfzeile ausgeben
\pagestyle{scrheadings}

% Listings sind in Java
\lstset{language=Java, float, captionpos=b, numbersep=5pt, frame=single, tabsize=2, aboveskip=1em, framerule=0.1pt, numbers=left, numberblanklines=false, basicstyle=\ttfamily}

% Alle Kopf- und Fußzeilen löschen
\clearscrheadings
\clearscrplain
\clearscrheadfoot

% Automatische Kopfzeile konfigureren mit Kapitelnamen und Abschnittsnamen
\automark[section]{chapter}
% Rechts unten immer Seitenzahl ausgeben
\ofoot[\pagemark]{\pagemark}
% In der Kopfzeile nur auf Seiten ohne neuen Kapitelanfang
\ihead[]{\leftmark}
\ohead[]{\rightmark}

%usepackages einbinden
\usepackage{float}
\usepackage{sidecap}
\usepackage{subfig}
\usepackage{pdfpages}
\usepackage{longtable}

% URLs darstellen
\usepackage{url}

% Querformat
\usepackage{lscape}

% Schusterjunge & Hurenkind
\clubpenalty = 10000
\widowpenalty = 10000
\displaywidowpenalty = 10000
\hbadness = 10000

% Satzspiegel vergrößern
\addtolength{\textheight}{1.6em}

% Kopfzeile nach unten verschieben
\addtolength{\voffset}{1.4em}

% markup definieren
\usepackage[normalem]{ulem}
\newcommand{\markup}[1]{\textbf{#1}}

% Einzug einstellen
\setlength{\parindent}{0.7em}

% Einzug einstellen
\setcapindent{1.5em}

%eigene befehle
\newcommand{\figref}[1]{(\figurename~\ref{#1},~S.~\pageref{#1})}
\newcommand{\eqnref}[1]{(Gleichung~\ref{#1},~S.~\pageref{#1})}
\newcommand{\defref}[1]{(Definition~\ref{#1},~S.~\pageref{#1})}
\newcommand{\gueref}[1]{(Vermutung~\ref{#1},~S.~\pageref{#1})}
\newcommand{\algref}[1]{(Algorithmus~\ref{#1},~S.~\pageref{#1})}
\newcommand{\tblref}[1]{(\tablename~\ref{#1},~S.~\pageref{#1})}
\newcommand{\chpref}[1]{(\chaptername~\ref{#1},~S.~\pageref{#1})}
\newcommand{\secpref}[1]{(\ref{#1},~S.~\pageref{#1})}
\newcommand{\appref}[1]{(\appendixname~\ref{#1},~S.~\pageref{#1})}
\newcommand{\coderef}[1]{(Code~\ref{#1},~S.~\pageref{#1})}

\usepackage{makeidx}

% pstricks Bilder erstellen
\usepackage{pstricks}
\usepackage{pst-node}

% Darstellung für Algorithmen, nummeriert anhand der Sektion, in der sie auftreten
\usepackage[section]{algorithm}
\usepackage{algorithmic}

% Ausgabe von Makros für Algorithmen verändern
\floatname{algorithm}{Algorithmus}
\renewcommand{\algorithmicrequire}{\textbf{Eingabe:}}
\renewcommand{\algorithmicensure}{\textbf{Ausgabe:}}
\renewcommand{\algorithmicwhile}{\textbf{solange}}
\renewcommand{\algorithmicfor}{\textbf{für}}
\renewcommand{\algorithmicdo}{\textbf{tue}}
\renewcommand{\algorithmicif}{\textbf{wenn}}
\renewcommand{\algorithmicthen}{\textbf{dann}}
\renewcommand{\algorithmicelse}{\textbf{sonst}}
\renewcommand{\algorithmicto}{\textbf{bis}}
\renewcommand{\algorithmicend}{\textbf{ende}}
\renewcommand{\algorithmicand}{\textbf{und}}
\renewcommand{\algorithmicor}{\textbf{oder}}
\renewcommand{\algorithmiccomment}[1]{// #1}

% Erscheinungsbild von Beweisen, Vermutungen etc festlegen.
\theoremstyle{theorem}
\newtheorem{theorem}{Theorem}[section]
\newtheorem{definition}[theorem]{Definition}
\newtheorem{corollary}[theorem]{Korollar}
\newtheorem{lemma}[theorem]{Lemma}
\newtheorem{guess}[theorem]{Vermutung}

% Deutsche Ausgabe
\renewcommand{\listalgorithmname}{Algorithmenverzeichnis}
\renewcommand{\nomname}{Abkürzungs- und Symbolverzeichnis}
\renewcommand{\lstlistingname}{Code}
\newcommand{\definitionname}{Definition}
\newcommand{\guessname}{Vermutung}

% Nummerierungsarten
\captionsetup{figurewithin=chapter,tablewithin=chapter}
\numberwithin{equation}{chapter}
\numberwithin{theorem}{chapter}
\numberwithin{algorithm}{chapter}
\numberwithin{table}{chapter}

% Punkte zw. Abkürzung und Erklärung
\setlength{\nomlabelwidth}{.20\hsize}
\renewcommand{\nomlabel}[1]{#1 \dotfill}
% Zeilenabstände verkleinern
\setlength{\nomitemsep}{-\parsep}

% ------------------------------------------------------------------------------
%
% ------------------------------------------------------------------------------
\author{Christopher Blöcker, Jan-Philip Loos}
\date{}
\title{E-Commerce Projekt}
\subtitle{Onlineshop mit Produktempfehlungsfunktion}

\makeindex
\makenomenclature

\usepackage{lipsum}
\usepackage[babel]{csquotes}

\usepackage{listings}
\lstloadlanguages{Haskell}
%\lstnewenvironment{code}
%    {\lstset{}%
%      \csname lst@SetFirstLabel\endcsname}
%    {\csname lst@SaveFirstLabel\endcsname}
%    \lstset{
%      basicstyle=\small\ttfamily,
%      flexiblecolumns=false,
%      basewidth={0.5em,0.45em},
%      literate={+}{{$+$}}1 {/}{{$/$}}1 {*}{{$*$}}1 {=}{{$=$}}1
%               {>}{{$>$}}1 {<}{{$<$}}1 {\\}{{$\lambda$}}1
%               {\\\\}{{\char`\\\char`\\}}1
%               {->}{{$\rightarrow$}}2 {>=}{{$\geq$}}2 {<-}{{$\leftarrow$}}2
%               {<=}{{$\leq$}}2 {=>}{{$\Rightarrow$}}2 
%               {\ .}{{$\circ$}}2 {\ .\ }{{$\circ$}}2
%               {>>}{{>>}}2 {>>=}{{>>=}}2
%               {|}{{$\mid$}}1               
%    }