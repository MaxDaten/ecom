\chapter{\index{Produktempfehlung}Produktempfehlung}
Produktempfehlungen können einem Benutzer auf Grundlage verschiedener Informationen gemacht werden. feature-basierte Emfpehlungen werden anhand der Ähnlichkeit von Produkten gemacht, association-basierte Empfehlungen anhand von als zusammengehörig definierten Produktkategorien.

\section{Feature-based}
Feature-basierte Produktempfehlungen werden auf der Grundlage dessen aus\-ge\-spro\-chen, was ein Benutzer bisher gekauft hat, was sich also in seiner \textsf{history} \coderef{code:User} befindet. Dabei wird berücksichtigt, dass einem Benutzer nur die Produkte empfohlen werden, die er auch tragen kann. \\
\\
Ein Benutzer kann ein Produkt tragen, wenn es in einer Größe verfügbar ist, in der er bereits ein anderes Produkt gekauft hat und wenn er die Anforderungen des Produktes erfüllt. Die ist der Fall, wenn die Anforderungen des Produktes bezüglich Stärke, Intelligenz, Geschicklichkeit und Ausdauer alle kleiner sind als die Attribute des Benutzers bezogen auf das jeweilige Merkmal. Sei \textsf{u} ein Benutzer und \textsf{ps} der Katalog an verfügbaren Produkten, dann können die Produkte, die von \textsf{u} getragen werden können, wie in \coderef{code:wearable} gezeigt, bestimmt werden.

\begin{lstlisting}[label=code:wearable,caption={Algorithmus zur Bestimmung von Produkten, die ein Benutzer tragen kann}]
wearable :: User -> [Product] -> [Product]
wearable u ps = attrOk u . sizeOk u $ ps
  where
    userSizes = foldr union empty . map productSizes . history
    sizeOk u  = filter (\p -> not . null . toList 
                       $ (productSizes p) `intersection` (userSizes u))
    attrOk u  = filter (\p -> (attributes u) < (productRequirements p))
\end{lstlisting}
\textsf{userSizes} berechnet zu einem gegebenen Benutzer die Menge der Größen, in denen er Produkte erworben hat. \\
\textsf{sizeOk} wählt aus einer Menge von Produkten diejenigen aus, die in einer Größe verfügbar sind, die zu einem gegebenen Benutzer passen.\\
\textsf{attrOk} schränkt die verfügbaren Produkte auf diejenigen ein, die der Benutzer aufgrund seiner und der Produktattribute tragen kann.
\\
Der in \coderef{code:wearable} beschriebene Algorithmus weicht von der tatsächlichen Implementierung ab, da im produktiven Einsatz aufgrund des Kontext monadisch gearbeitet werden muss. Das hier gezeigte Verfahren ist unabhängig davon korrekt und dient der Verdeutlichung der Idee für feature-basierte Produktempfehlungen\\
\\
Sind die von einem Benutzer tragbaren Produkte ausgewählt, so wird zwischen diesen und den vom Benutzer bereits gekauften Produkten paarweise die Ähnlichkeit berechnet, um Empfehlungen zu generieren. Im folgenden wird zunächst be\-schrie\-ben, wie die Ähnlichkeit zwischen zwei Produkten bestimmt werden kann.\\
\\
Um die Ähnlichkeit zwischen zwei Produkten zu bestimmen, findet zunächst eine Projektion auf ihre Boni \textsf{productAttributes} statt, welche als Vektoren im euklidischen Raum $\mathbb{R}^4$ interpretiert werden können. Eine Erweiterung auf weitere Attribute ist problemlos möglich und würde für $n$ Attribute in einer Projektion in den euklidischen Raum $\mathbb{R}^n$ resultieren. Als Ähnlichkeitsmaß dient die euklidische Norm im jeweiligen Raum, welche definiert ist als
\begin{equation}
  \left|\left| x \right|\right| = \sqrt{\sum_{i=1}^n x_i^2}.
\end{equation}
Der Algorithmus zur Berechnung der Ähnlichkeit von Produkten ist in \coderef{code:dist} aufgezeigt.

\begin{lstlisting}[label=code:dist,caption={Algorithmus zur Berechnung der Ähnlichkeit von Produkten}]
attribs :: [Attributes -> Double]
attribs = map (fromIntegral .)
              [unStr . str, unInt . int, unDex . dex, unSta . sta]

(-?) :: Product -> Product -> Double
p1 -? p2 = (attr p1) `dist` (attr p2)
  where
    dist x y = sqrt . foldr (+) 0 . map (**2.0) $ zipWith (-) x y
    attr     = \p -> map ($ productAttributes p) attribs
\end{lstlisting}
\textsf{attribs} führt die Projektion von Produkten in den $\mathbb{R}^4$ durch. \\
\textsf{-?} bestimmt die Ähnlichkeit zwischen zwei Produkten.\\
\\
Da die Ähnlichkeit von Produkten definiert ist als die Norm der Differenz ihrer Eigenschaftsvektoren, projiziert in den $\mathbb{R}^n$ (hier $n = 4$), ist diese symmetrisch und es gilt:
\begin{equation}
  \bigwedge_{p_1, p_2 \in P} \delta\left(p_1, p_2\right) = \delta\left(p_2, p_1\right)
\end{equation}
mit Produkten $p_1, p_2$ und der Menge aller Produkte $P$. Dabei sei $\delta$ identisch zum in Code \ref{code:dist} beschriebenen Infix-Operator \enquote{\textsf{-?}}.\\
\\
Die durch \textsf{-?} gegebene Ähnlichkeit zwischen Produkten wird verwendet, um die Produktempfehlungen für einen Benutzer zu sortieren. Sei \textsf{Set bought} die Menge der Produkte, die ein Benutzer bereits gekauft hat und \textsf{ps} die Menge der Produkte, die ebendieser Benutzer tragen kann. Die Berechnung der Produktempfehlungen, aufsteigend sortiert nach der Ähnlichkeit zu einem bereits gekauften Produkt, ist in \coderef{code:fb-suggestions} aufgezeigt.

\begin{lstlisting}[label=code:fb-suggestions,caption={Algorithmus zur Generierung feature-basierter Produktempfehlungen}]
similarProducts :: [Product] -> [Product] -> [Product]
similarProducts bought ps = 
    let notBought = filter removeBought ps
        pairs     = [(x -? y, y) | x <- bought, y <- notBought]
    in nub . map snd . sort $ pairs
  where
    removeBought = \p -> (productId p) `notElem` (map productId ps)
\end{lstlisting}


\section{Association-based}
Association-basierte Empfehlungen werden auf der Grundlage fest verknüpter Kategorien ausgesprochen, welche durch die in \coderef{code:Association} beschriebene Datenstruktur realisiert werden. Somit wird jedem Produkt eine Menge assoziierter Produkte zugeordnet. Sei \textsf{p} ein Produkt, \textsf{ps} die Menge aller verfügbaren Produkte und \textsf{as} die Menger der definierten Assoziationen. Der in \coderef{code:ab-suggestions} gezeigte Algorithmus dient der Berechnung assoziations-basierter Produktempfehlungen.

\begin{lstlisting}[label=code:ab-suggestions,caption={Algorithmus zur Generierung association-basierter Produktempfehlungen}]
associationBased :: Product -> [Product] -> [Association] -> [Product]
associationBased p ps as =
  let cats = foldr union empty . map assocedCategories . asc $ as
  in filter (not . null . intersection cats . productCategories) ps
  where
    asc = filter (\a -> assocCategory a `member` productCategories p)
\end{lstlisting}
Der in \coderef{code:ab-suggestions} beschriebene Algorithmus weicht erneut von der tatsächlichen Implementierung ab, da diese monadisch arbeiten muss. Das Verfahren ist unabhängig davon korrekt und illustriert das prinzipielle Vorgehen.