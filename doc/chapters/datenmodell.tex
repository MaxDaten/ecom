\chapter{\index{Datenmodell}Datenmodell}
Das dem Empfehlungssystems zugrunde liegende Datenmodell \textsf{EcomState} besteht aus einem Produktkatalog, Assoziationen und Benutzern. Dabei beschreibt \textsf{IxSet a}, dass es sich um eine indexierte Menge des Datentyps \textsf{a} handelt.

\begin{lstlisting}[label=EcomState,caption=Struktur von \textsf{EcomState}]
data EcomState = EcomState { catalog :: IxSet Product
                           , assocs  :: IxSet Association
                           , users   :: IxSet User
                           }
\end{lstlisting}


\section{\index{Produkte}Produkte}
Der Produktkatalog besteht aus einer Menge von Produkten. Ein Produkt ist eindeutig durch seine \textsf{ProcuctId} (Z. 1) gekennzeichnet. Jedes Produkt verfügt über einen Namen \textsf{ProductTitle} (Z. 2) und einen Slot \textsf{ProductSlot} (Z. 3), der beschreibt, an welcher Körperstelle das Produkt getragen wird. Produkte können mehreren Kategorien angehören sowie in verschiedenen Größen bzw. Farben verfügbar sein, die wird durch modelliert, dass jedes Produkt über eine Menge (\textsf{Set}) der jeweiligen Eigenschaften verfügt (Z. 4-6). Die Anforderungen eines Produktes an den Träger werden durch Attribute beschrieben (Z. 7), die Boni, die ein Produkt seinem Träger bringt, sind ebenfalls durch Attribute beschrieben (Z. 8). Zuletzt verfügen Produkte über eine Produktbeschreibung \textsf{ProductDecription} (Z. 9).

\begin{lstlisting}[label=Product,caption=Struktur von \textsf{Product}]
data Product = Product { productId           :: ProductId
                       , productTitle        :: ProductTitle
                       , productSlot         :: ProductSlot
                       , productCategories   :: Set ProductCategory
                       , productSizes        :: Set ProductSize
                       , productColors       :: Set ProductColor
                       , productRequirements :: Attributes
                       , productAttributes   :: Attributes
                       , productDescription  :: ProductDescription
                       }
\end{lstlisting}
Attribute enthalten Informationen über Stärke, Intelligenz, Geschicklichkeit und Ausdauer. Im Bezug auf Produkte werden Attribute einerseits verwendet, um zu be\-schrei\-ben, welche Anforderungen sie an ihren Träger stellen und andererseits, welche Boni sie bie\-ten.

\begin{lstlisting}[label=Attributes,caption=Struktur von \textsf{Attributes} und zugehörige Zugriffsfunktionen]
data Attributes = Attributes { str :: Strength
                             , int :: Intelligence
                             , dex :: Dexterity
                             , sta :: Stamina
                             }

newtype Strength     = Strength     Int
newtype Intelligence = Intelligence Int
newtype Dexterity    = Dexterity    Int
newtype Stamina      = Stamina      Int

unStr :: Strength -> Int
unStr (Strength     s) = s

unInt :: Intelligence -> Int
unInt (Intelligence i) = i

unDex :: Dexterity -> Int
unDex (Dexterity    d) = d

unSta :: Stamina -> Int
unSta (Stamina      s) = s
\end{lstlisting}
Die in Code \ref{Attributes} gezeigte Struktur wurde gewählt um Haskells Möglichkeiten der Typsicherheit zu nutzen und eine Unterscheidung zwischen den Typen von Stärke, Intelligenz, Geschicklichkeit und Ausdauer zu machen, sodass diese nicht kompatibel sind. Dennoch können die eingewickelten Daten mit Hilfe der Funktionen \textsf{unStr}, \textsf{unInt}, \textsf{unDex} und \textsf{unSta} ausgewickelt werden, um auf dieser Grundlage Berechnungen durchzuführen.


\section{Kategorieverknüpfung}
Zur Umsetzung von Assoziationsbasierten Produktempfehlungen sollen die Kategorien verwendet werden, denen Produkte zugeordnet sind. Für jede Kategorie \textsf{assocCategory} kann eine Menge von Kategorien \textsf{assocedCategories} definiert werden, die mit dieser Kategorie assoziiert sind.

\begin{lstlisting}[label=Association,caption=Struktur von \textsf{Association}]
data Association = Association
                 { assocCategory     :: ProductCategory
                 , assocedCategories :: Set ProductCategory
                 }
\end{lstlisting}


\section{Benutzer}
Benutzer werden identifiziert durch einen Namen \textsf{username} und verfügen über Attribute. Außerdem ist für jeden Benutzer ein Verlauf \textsf{history} an gekauften Produkten bekannt, auf dessen Grundlage feature-basierte Produktempfehlungen ausgesprochen werden.
\begin{lstlisting}[label=User,caption=Struktur vom \textsf{User}]
data User = User { username   :: Text
                 , history    :: [Product]
                 , attributes :: Attributes
                 }
\end{lstlisting}